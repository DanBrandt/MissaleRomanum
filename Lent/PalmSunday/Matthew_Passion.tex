\documentclass{../../missal-sheet}

\begin{document}

\chapter*{Evangelium Passionis et Mortis domini Secundum Matth\'{\ae}um}

\begin{center}
 Matthew 26, 36-75; 27, 1-60
\end{center}

\begin{rubricbox}
{\color{red} The Gospel procession having taken place, the clerics assemble themselves on the Gospel side, facing liturgical north. The Passion Narrative is chanted, with \textit{four} parts contributing: The Chronicler (symbolized by a letter `C'), Christ (symbolized by a `\maltese'), a singular Synagogue part (symbolized by a `S'), and a plural Synagogue part, known as the `Turba', literally meaning `crowd' (symbolized by a `T').  The Chronicler begins the chanting of the Passion Narrative. For ease of chanting, the singular Synagogue part has been tranposed down a perfect fourth from its original setting.}
\end{rubricbox}

% Page 1:

\gresetinitiallines{1}
\gregorioscore{matthew_passion_1}

% Page 2:

\gresetinitiallines{0}
\gregorioscore{matthew_passion_2}

% Page 3:

\gresetinitiallines{0}
\gregorioscore{matthew_passion_3}

% Page 4:

\gresetinitiallines{0}
\gregorioscore{matthew_passion_4}

% Page 5:

\gresetinitiallines{0}
\gregorioscore{matthew_passion_5}

% Page 6:

\gresetinitiallines{0}
\gregorioscore{matthew_passion_6}

% Page 7:

\gresetinitiallines{0}
\gregorioscore{matthew_passion_7}

% Page 8:

\gresetinitiallines{0}
\gregorioscore{matthew_passion_8}

% Page 9:

\gresetinitiallines{0}
\gregorioscore{matthew_passion_9}

% Page 10:

\gresetinitiallines{0}
\gregorioscore{matthew_passion_10}

% Page 11:

\gresetinitiallines{0}
\gregorioscore{matthew_passion_11}

\end{document}